\section{Physikalische Grundlagen}
\subsection{Entstehung und Zerfall von Myonen}
Im Experiment werden Myonen betrachet, die durch kosmische Höhenstrahlung erzeugt werden. Diese besteht überwiegend aus hochenergetischen Protonen, die dann in der oberen Atmosphäre mit Molekülen wechselwirken. Dabei finden folgende Reaktionen statt:
\begin{align}
	&p+p \rightarrow p+n+\pi ^{+} \\
	&p+n \rightarrow p+p+\pi ^{-}\\
	&p+p \rightarrow  p+ \Lambda + K^{+}
\end{align}

\noindent Die entstandenen Pionen und Kaonen sind instabil und zerfallen in Myonen und Neutrinos. Da die kosmische Strahlung überwiegend positiv geladen ist, werden insgesamt mehr positive Teilchen erzeugt und das Verhältnis von positiven zu negativen Myonen beträgt $\approx 1,25$. Sie zerfallen spätestens wenn sie zur Ruhe kommen über die schwache Wechselwirkung (WW) und haben eine durchschnittliche Lebensdauer von $2,19 \mu s$. Der Zerfall erfolgt über folgende Reaktionen:

\begin{align}
	&\mu ^{-} \rightarrow e^{-} + \bar{\nu} _{e} +\nu _{\mu} \\
	&\mu ^{+} \rightarrow e^{+} + \bar{\nu} _{\mu} +\nu _{e}
\end{align}
Da die $\mu ^{-}$ auch von einem Kern eingefangen werden können gibt es für diese Myonen einen zusätzlichen Zerfallskanal. Die eingefangenen Myonen bilden mit dem Atom ein myonisches Atom, bei dem das Myon vom Kern absorbiert wird. Dabei zerfällt es über einen inversen Betazerfall und das enstandene Atom zerfällt über einen Betazerfall wieder zum ursprünglichen Atom. Somit ergibt sich für die Lebensdauer von $\mu ^{-}$:
\begin{equation}
	\frac{1}{\tau} = \frac{1}{\tau _0} + \frac{1}{\tau _C}
\end{equation}
Wobei $\tau_C$ die Lebensdauer beim Myon-Einfang beschreibt.
\newline
Im Experiment wird eine Mischung aus positiven und negativen Myonen detektiert. Man erhält folgendes Zerfallsgesetz:
\begin{equation}
\label{eq:7}
	N(t)=N(\mu ^{-},t_0)\cdot \exp{\frac{t-t_0}{\tau_0}} \cdot \exp{\frac{t-t_0}{\tau_C}} + N(\mu ^{+},t_0) \cdot \exp{\frac{t-t_0}{\tau_0}}+BG
\end{equation}

\subsection{Paritätsverletzung}
Die durch den Zerfall entstandenen Elektronen weisen eine asymmetrische Winkelverteilung auf. Dies kommt dadurch zustande, dass die WW proportional zur Geschwindigkeit $\beta$ an linkshändige und proportional zu $1-\beta$ an rechtshändige massive Teilchen koppelt. Für Antiteilchen gelten die umgekehrten Relationen. Beim $\mu$-Zerfall gleicht sich der Spin der Neutrinos aus, weshalb der Spin des $e$ aus Erhaltungsgründen gleichgerichtet zum Spin des $\mu$ ist. Die verschieden starke Kopplung der schwachen WW an links- und rechtshändige Teilchen hat zur Folge, dass rechtshändige Elektronen (Helizität=+1) seltener ausgesendet werden als linkshändige(Helizität=-1). Für den Zerfall von $\mu$ findet man für die räumliche Verteilung $N(\phi) \propto 1+A cos(\phi) $ mit dem von der Positronenenergie abhängigen Asymmetrieparameter A. Für den verwendeten Aufbau erwartet man A=0.23. Da die Myonen ebenfalls über Zerfallsprozesse der schwachen WW entstehen existiert eine Asymmetrie in der Zerfallsrichtung, die eine Polarisation der kosmische Myonen bewirkt. Die Polarisation P für Pionen als Primärteilchen beträgt $P = 0,33$ und für Kaonen $P = 0,54$. Der Versuch ist so aufgebaut, dass lediglich hochenergetische Elektronen detektiert werden, dh. beide Neutrinos werden im Winkel von$180^{°}$ zur Flugrichtung des Elektrons emittiert. Das Myon besitzt ein magnetisches Moment $\vec{\mu}$, das sich wie folgt berechnen lässt:
\begin{equation}
	\vec{\mu} _{\mu} = g_{\mu} \cdot \mu ^{Bohr} _{\mu} \cdot \vec{s}
	\quad
	\text{mit Bohrschem Magneton}
	\quad 
	\mu ^{Bohr} _{\mu} = \frac{e \hbar}{2 m_{\mu}}
\end{equation}In einem externen Magnetfeld $\vec{B}$, das senkrecht zum Spin $\vec{s}$ des Myons steht, führt sein magnetisches Moment eine Larmorpräzession mit der Frequenz
\begin{equation}
	\omega_{Larmor}
	=
	\frac{g\cdot \mu ^{Bohr} _{\mu}\cdot B}{\hbar}
\end{equation}
aus. Beobachtet man den zeitlichen Verlauf der Zerfallasymmetrie, so kann man die Spinpräzession verfolgen. Beim Zerfall von $\mu ^{+}$ wird, wie oben bereits erklärt, das $e^{+}$ bevorzugt in Richtung des Myonspins und somit auch in Richtung des magnetischen Moments ausgesandt.