\section{Discussion}
At first several LabView modules where programmed. Those primarily set the frequency of the frequency generator (Fig.\,\ref{fig9}, left) and read the output of the lock-in amplifier (Fig.\,\ref{fig10}, left). Since the desired output was the amplitude $A$ and the phase $\phi$ the generated outputs from the amplifier ($A \cos \phi$ and $A \sin \phi$ ) where converted, as seen in (Fig.\,\ref{fig10}, left).\\
Then resonance curves where obtained. The measurement of the amplitude and the phase of around the ground frequency can be seen in (Fig.\,\ref{fig1}). The fit can be seen in (Fig.\,\ref{fig2}). It is in good agreement with our expectation, given a reduced $\chi^2$ of 1.33.\\
We expected to find the ground resonance frequency at $\nu_0 = 274 \text{Hz}$. Through our measurements we found it to be at roughly $\omega_R = 319.436 \pm 0.008$Hz. This discrepancy can have several reasons, most likely the apparatus differs slightly from the values used in the calculation.\\
Before getting a reasonable measurement we experienced several difficulties with the equipment. At the end we used a different lock-in amplifier and a different vibrating reed to achieve the above mentioned results.\\
Thus there are several systematic errors that come in play. Mainly the manufacturing of the reed and the way it vibrates does not correspond exactly to the given data, since multiple setups are in use. In addition there is lots of noise. Since we found expected the desired amplitude to be of the order of $\mu V$, noise peaks are substantial. Through a measurement of the phase the resonance peaks can be identified through the characteristic phase shift.  
\\
In part II the temperature dependence of the resonance frequency was measured. Again different LabView modules where programmed to suit the purpose of setting the voltage of the Peltier element and read the temperature form the Pt-1000 thermometer (Fig.\,\ref{fig8} to (Fig.\,\ref{fig11})). As can be seen in the temperature curves in (Fig.\,\ref{fig13}, left) the thermal equilibrium was not yet reached when the measurements began. Our initial findings suggested that thermal equilibrium would be reached in a matter of 10 minutes. As we had more time to spare during the night measurement we doubled the waiting time to 20 min. Our data suggests that equilibrium was reached after less than 5Hz into our measurement. Thus implying a waiting time of about 30-40 minutes.\\

