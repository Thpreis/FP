\section{Diskussion}
Eine Möglichkeit um die systematischen Fehler zu verringern wäre eine bessere Nachpulskorrektur. Bei der verwendeten Methode wird nämlich vom untersten Layer angefangen auf Nachpulse zu prüfen und, sobald einer gefunden wird, wird die Routine abgebrochen. Zur Verbesserung sollte die Routine nicht abbrechen und auch die Layer darüber auf Nachpulse prüfen.
Durch die Auswertung des grö{\ss}eren Datensatzes \ref{fig:Abbildung 4} wird klar, dass in diesem Versuch die Statistik zwar eine Auswirkung auf die Ergebnisse hat, aber auch dass die gro{\ss}en systematischen Fehler sich wesentlich relevanter auf die Ergebnisse auswirken(vergleiche dazu die Veränderung der Werte durch das Ändern der Fitrange)s. Die Auswirkung der systematischen Fehler könnte man möglicherweise durch die Verwendung eines passenderen Modells verringern.
Zusätzlich ist noch zu sagen, dass bei unserem Versuch nur sechs der insgesamt acht Szintillatoren genutzt werden konnten und somit allein durch den Aufbau schon eine Einschränkung vorlag. Abschlie{\ss}end kann man sagen, dass der Versuch gut geeignet ist, um qualitative Erkenntnise zu erlangen, jedoch sollte man den Messwerten nicht zu viel Bedeutung beimessen, da auch die Fehler dieser relativ gro{\ss} sind.
