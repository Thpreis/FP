\section{Theory}
\subsection{The muon}
The muon is an elementary particle. In the Standard Model of particle physics it belongs to the group of leptons. It has an electric charge of -1e and spin $s = \frac{1}{2}$, but is approximately 207 times heavier than the electron ($m_{\mu} = 105.7$MeV / c$^2$) [\ref{muon}].\\
Cosmic muons are produced from decaying pions and kaons, which in turn are produced by energetic protons coming from primary cosmic rays. The production of muons from decaying pions can amongst others occur as follows
\begin{equation}
\pi^+ \rightarrow \mu^+ + \nu_{\mu}
\label{piPlus}
\end{equation}
\begin{equation}
\pi^- \rightarrow \mu^- + \bar{\nu_{\mu}}.
\label{piMinus}
\end{equation}
The primary cosmic radiation is positively charged. Hence the decay into positive muons is favourable.\\
Muons have an average lifetime of $\tau = 2.19\mu$s [\ref{manual}]. A muon decays into an electron and two neutrinos via the weak interaction 
\begin{equation}
\mu^- \rightarrow e^- + \nu_{\mu} = + \bar{\nu_{e}}
\end{equation}
\begin{equation}
\mu^+ \rightarrow e^+ + \nu_{e} = + \bar{\nu_{\mu}}.
\end{equation}
With the average lifetime $\tau$ the decay follows an exponential law
\begin{equation}
N(t) = N(t_0) e^{- \frac{t - t_0}{\tau}}.
\end{equation}
Both positive and negative muons can decay as free particles or by capturing an electron and thus forming the so called muon atom. Negative muons can however also be captured by atomic nuclei. This effectively reduces the lifetime of negative muons to
\begin{equation}
\frac{1}{\tau} = \frac{1}{\tau_0} + \frac{1}{\tau_c},
\end{equation}
with $\tau_0$ and $\tau_c$ being the muon lifetime and the lifetime of a captured muon respectively.\\
With a measurement of the lifetime $\tau_0$ and knowledge of the muon mass $m_{\mu}$ the Fermi constant $G_F$ can be determined
\begin{equation}
G_F^2 = \frac{192 \cdot \pi^3 \cdot \hbar}{\tau_0 \cdot (m_{\mu} c^2)^5}.
\end{equation}


\subsection{Parity violation}
A parity transformation refers to the flip in the sign of spacial coordinates [\ref{parity}]. Applying a parity transformation to a particle with momentum \textbf{p} and spin \textbf{s} results in a change of sign in \textbf{p} and an unchanged spin.\\
The relative direction between momentum and spin is captured by the concept of helicity. If both momentum and spin point in the same direction, one speaks of a helicity of +1 (right-handed). If they point in different directions the helicity is -1 and one speaks of a left-handed particle.\\
Now, the weak interaction couples only to the left-handed particles and the right-handed part of anti-particles. Assuming that the particle is massless, such that helicity and charlity coincide. As described above, cosmic muons are produced by decaying pions. Pions have spin 0, thus the spin has to be conserved in its decay. This leads to two different configurations. One, where the helicity of the products (the muon and the muon-neutrino, cf. Eq. \ref{piMinus}, \ref{piPlus}) is +1 and one, where it is -1. Since the neutrino is almost massless, the weak interaction suppresses the production of left-handed neutrinos. This gives the muons a preferred spin direction, thus violating parity. By measuring the polarisation of incoming muons the asymmetry in the distribution of left- and right-handed muons can be determined. Thus giving proof of the parity violation of the weak interaction.\\

\subsection{Magnetic moment}
The magnetic moment of a muon is given by
\begin{equation}
\mu_{\mu} = g_\mu \cdot \mu_\mu^{Bohr} \cdot \mathbf{s}
\end{equation}
where the Bohr-magneton is given as $\mu_\mu^{Bohr} = \frac{e \hbar}{2 m_\mu}$. If the spin of the muon is perpendicular to an external magnetic flux \textbf{B}, the magnetic moment of the muon performs a Larmor precession with
\begin{equation}
w_{Larmor} = \frac{g \cdot \mu_\mu^{Bohr} \cdot \mathbf{B}}{\hbar}.
\end{equation}
This precession is observable by measuring the time dependence of the asymmetry in the decay of the cosmic muons. This is made possible by the fact that in a $\mu^+$-decay the positron is mostly emitted in the direction of the muon spin [\ref{manual}]. 


