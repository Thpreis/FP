\section{Discussion and Summary}
Our final result for the lifetime of muons ($\tau_0=(2.28\pm0.05_{\text{stat}}\pm0.22_{\text{sys}})\mu\text{s}$) deviates from the literature value ($\tau = 2.197 \mu$s \ref{manual}) by only $0.35\sigma$. The Fermi constant obviously deviates from the literature value by the exactly the same margin ($0.35\sigma$). From the polarisation measurement we got $	\mu_{\mu}^{\text{Bohr}} = (3.82\pm0.26_{\text{stat}}\pm0.39_{\text{sys}})\cdot10^{-26}\frac{J}{T}
$ which deviates from the expected value of $4.490\cdot 10^{-26}\frac{\text{J}}{\text{T}}$ by $1.4\sigma$. Given the low statistics on this part of the experiment this seems to be as good as one can expect the result to be. It is in the right order of magnitude though which does show that the muons reaching the earth are polarized and decay in a parity violating way. 

In the experiment statistics play a big role so by doing longer measurements and taking more data one could probably improve the results. Lowering the statistical error might not seem to be as relevant compared to the big systematic errors, but by doing so one might be able to find an even better model, that takes into account even more of the systematics, than the one used in this analysis. The fact, that the fitrange had such a significant impact on the outcome of the experiment shows, that some systematic effect has not been taken into account yet. It would also be very interesting to analyse whether or not the after pulses depend on the energy or the energy loss of the muons. The events that have been used to get the afterpulse spectrum where from muons that flew further than the ones that could have created a false event, so the energy loss of the muon might be different in this case. 