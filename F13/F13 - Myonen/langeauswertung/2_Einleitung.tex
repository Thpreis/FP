\section{Introduction}
The muon was first discovered in 1936 when researchers at Caltech observed a particle that came from cosmic radiation and that, when passing a magnetic field, curved less than an electron but more than a proton and was assumed to have the same charge as the electron. This particle was first thought to be the so-called Yukawa particle, which today is known as a pion, because of the similar mass. Nowadays muons play a significant role in particle physics, as it is - as far as we know - an elementary particle with a mass small compared to typical energies in modern particle physics experiments. For example muons played a role in the discovery of the higgs boson (\ref{higgs_cms},\ref{higgs_atlas}), as one of the clearest decay channels for the higgs boson is into two Z bosons which in turn decay into four leptons. Muons were also used in the Rossi-Hall experiment (1941) to confirm time dilation as predicted by Einstein's theory of special relativity. 
What also makes them interesting, is the fact that the production as well as the decay of cosmic muons violates the parity symmetry that was long thought to be a universal symmetry of the universe. This parity violation in the weak interaction as well as other basic properties of the muon will also be analysed in this experiment. \ref{muon}