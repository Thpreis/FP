Test
The oszilloscope is connected to the output of the PMs of the scintillators and signals of roughly 50 to 500mV can be observed, whereas really big pulses dont occur very frequently. In Fig. \ref{Fig_1} one can see a typical pulse of the scintillator. Changing the terminating resistor from 50$\Omega$ to 1M$\Omega$ does not have a huge impact on the observed pulses, it does give a little overshoot after the pulse gets back to zero though (See Fig. \ref{Fig_2}). 

\begin{enumerate}
\item How are photons detected in a photomultiplier?\\
\item How are the photons generated in the scintillator?\\
The answers to the rist two questions can be found in section 3 from above.
\item How much energy deposit relativistic particles such as muons from cosmic radiation in a scintillator?\\

\item Which impact has the terminator on the waveform?\\
\item How do pulses from photomultiplier tubes look like?\\
\item What determines the height of these pulses?\\
\end{enumerate}

Next the thresholds are changed and a lower threshold (absolute value, since the pulses are negative) yields a higher rate but also seems to give an output signal if there is no clear signal. A higher threshold on the other hand yields a lower rate of very high peaked signals. Thresholds of around -10mV seemed to be a good compromise. The pulse heigths still varied between 50 to 500mV. The signal of the discriminator was slightly shifted from the scintillatorsignal by around 80ns and had a width of around 102ns. 

The scintillators have a width of $32.5\pm0.5$cm and a length of $82\pm2$cm. The length was not easily measurable since it was not clear where the scintillator ended and the readout started. The top of the photomultipliers where measured to be at $9.5 \pm 0.5; 85.0 \pm 0.5; 76.0 \pm 0.5; 66.0 \pm 0.5; 56.5 \pm 0.5; 47.5 \pm 0.5; 37.5 \pm 0.5; 28.0 \pm 0.5$ above ground. 

The couting rates were found to be around 100Hz for the thresholds used as above. For measurements that had the layer above and below as a reference, the detection efficiency was measured (assuming, that a signal in the two reference plains at the same time that does not give a signal in the measured layer means that a signal should have been observed but hasn't) to be 96.6\% for scintillator 1, 96.2\% for scintillator 2, 96.1\% for scintillator 3 and 91.6\% for scintillator 4. The thresholds where kept at -10mV since changing them did not improve the efficiency for scintillator 4. 

After this the rates drastically changed, so that we recalibrated the thresholds to -10mV for scintillator 0, -10mV for scintillator 1, -15mV for scintillator 2, -17mV for scintillator 3, -15mV for scintillator 4 and -30 for scintillator 5.
This gave us rates of 124.5Hz for scintillator 0, 128.4Hz for scintillator 1, 119.8Hz for scintillator 2, 140.3Hz for scintillator 3, 122.4Hz for scintillator 4 and 113.2Hz for scintillator 5. With these we got efficiencies of 97.5 (sc.1), 96.5 (sc.2), 94.9 (sc.3), 95.4 (sc.4) respectively. 