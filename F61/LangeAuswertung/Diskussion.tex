\section{Discussion - Tobias Abele}
In the first two parts we used a setup being isolated by Styrofoam. This of course isolates, but not as good as one would expect an isolation for an advanced practical course.\\
One had to use several plates of Styrofoam to isolate the setup, which provides much more possibilities for exchange of thermal energy than a complete box of Styrofoam. The room temperature influenced the magnetic field we had to adjust, which made the already quite difficult adjustment of the field even less accurate or stable. Turning the adjustment wheel always in one direction changed the magnetic field in long-time range either up or down, but for small adjustments the field sometimes went up and then down, which made corrections for the working frequency after each measurement due to thermal exchange complicated.
Either more adjustments or simply a better thermal isolation of the whole setup for the first two parts would improve our measurements significantly. Additionally, it would have helped, if one would have known what changing the adjustment wheel really changed and why the values sometimes went up or down when readjusting the field.
An improvement for the used program LabView would be a change in the step size of switches of the used Minispec 20 and a measurement of more data points to improve the fits and lower statistical errors.\\
Some reference values would improve the measurement, too.\\
\\
\noindent
When measuring the chemical shifts, some changes of peaks appeared due to readjustments of the magnetic field because of exchange of thermal energy. But all in all, the measurements of this part were fine. Only the noise of the pressure air one could improve with a motorised setup spinning the glass tubes of the different probes electrically and not mechanically due to air.\\
\\
For the third part, the imaging part of this experiment, we used another setup, the Bruker NMR analyzer mq7.5, which was isolated thermally very well. Some magnetic field non-linearities skewed some of our pictures. Another problem was the very small tube for own probes, which provided us some extra work when preparing the measurement with the apple core. But solving this problem would result in a new and bigger NMR analyser, which probably is too expensive. 
An improvement as well could be another probe, which refers a little more to the medical use of the NMR technique such as probes of organisms.\\
\\
Another aspect is the manual, which was fine until part three. Here it was too detail-obsessed and only very confusingly explained the theoretical principle of this part. A greater depth in the part of the two-dimensional data acquisition instead would ease the experiment for other groups in the future.\\
\\
To summarise, the experiment was very insightful and interesting to get to know the widely used NMR technique better and deepen knowledge connected with the theory behind it. One could just improve the first part by improving the setup and the second part by cutting and improving the manual as discussed above.\\
