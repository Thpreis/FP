
\section{Discussion - Thimo Preis}
The setup in the first two parts of the experiment was not isolated very well thermally such that the $B_0$ magnetic field was very sensitive to work with. The skew available for varying $B_0$ is furthermore very unprecise and does not increase or decrease $B_0$ consistently with a consistent rotating motion. It was therefore very difficult to calibrate the working frequency, which again is only necessary due to the ineffective thermal shielding with styrofoam, consistently after every measurement due to the unprecise adjustability of $B_0$. One should therefore improve the thermal shielding of the Minispec p20 magnet in order to render the ongoing calibration of the working frequency after every measurement unnecessary in the boundaries of the setups precision. One should furthermore increase the number of possible data points for the used fit of $T_1$ and $T_2$ in order to increase the precision of the measured values. This can easily be achieved by expanding the available LabView program to guarantee more than ten data points to be accounted for in the fit. One could furthermore refine the step size of the switches of the Minispec p20 electronic unit in order to account for more data points in the steeply falling/rising range of the exponential decay/rise. In total it would improve the precision of the measured results, if one would simply increase the number of taken data points during the measurement and also increase the number of measurements and thus improve on the fits in order to average out the huge systematic errors due to the sensitivity of the apparatus on the chosen peak window and on the thermal shielding.  One should furthermore be provided with theoretical values of the relaxation times in order to guarantee a successfull discussion of the measured values and a prompt error evaluation during the experiment for measured values way off the theoretical values. These error sources had a higher impact on the first part of the measurement than on the second part concerned with the chemical shift. The systematic errors averaged out due to the structure of the measurement relying only on differences between the peaks.\\
\\
The third part of the experiment went very smooth as the Bruker NMR analyzer mq7.5 was thermally shielded very well (compare \ref{sec:4}) and had a high image resolution, only the intrinsic magnetic field non-linearities distorted our images a bit. By trying out the different slicing methods we could nevertheless 
achieve images of high resolution from which we could draw important physical conclusions.\\
The instruction manual \cite{manual} gives a clear introduction to the NMR concepts. The third part concerned with "Imaging with NMR" is to in depth for this experiment and misses to convey the quintessential points of this method. One should therefore cut down the explanation of the actual Fourier transformation of the NMR signal and in contrast should go more in depth in explaining the actual two dimensional data aquisition and on how the Helmholtz coils were set up in order to guaruantee linear gradient fields.
\\
Altogether this experiment is very instructive in conveying the import NMR methods used throughout the medical and food sector in an engaging way. Only the first part of the experiment could be improved in order to increase the precision of the measured results as discussed above.