\section{Discussion}
The results obtained were good enough for determining setup characteristics, but making a master-flat-field with the use of a white blanket illuminated only by an orange lightsource is not sufficient for actual observations of the night sky. Here one should go over to making sky flat-fields orienting the telescope along the horizon.\\
The execution of the observational part of the measurement was then again very easy, as our tutor provided us with prewritten scripts.\\
The deviation of our CMD from the literature one can partially be explained by inaccuracies in the zeropoint determination. These were due to us not being able to exactly define a region around enough reference stars, where no interference of adjacent stars could be guaranteed. Said inaccuracies would cause a small horizontal shift in the Colour Magnitude Diagramm. Varying the metallicty results in a horizontal offset of the isochrones in the red giant branch. To get better results, the zeropoint calibration would have to be determined more accurately, e.g. by using more reference stars. Furthermore, better results could most definately be achieved by using a computed fit rather than the manual fit that we performed. This would probably counteract our misjudgement, which was to decrease the shift rather than decreasing the metallicity, and result in obtaining a better fitting result. Furthermore, determining the PSF was really difficult to do since the images of the BS90 provided by HST showed an artefact in the upper right corner due to very bright stars. One could optimize this procedure by letting the programm select all possible stars, excluding all the ones with exceedingly high or low surface brightness, in order to obtain a good PSF, because the main error was due to us not being able to exactly select the stars on the image with the crosshair provided.\\
